\chapter[Umgang mit Literatur]{Umgang mit Literatur und anderen Quellen}
\label{cha:Literatur}


\paragraph{Anmerkung:}
Der Titel dieses Kapitels ist für die Kopfzeile (absichtlich) zu
lang geraten; in diesem Fall kann man in der {\tt
chapter}-Anweisung von \latex als optionales Argument \verb![..]! einen verkürzten Text für die
Kopfzeile angeben:
\begin{verbatim}
  \chapter[Umgang mit Literatur]
          {Umgang mit Literatur und anderen Quellen}
\end{verbatim}

\section{Allgemeines}

Für die Gestaltung der Literaturverweise im Text und der
Quellenangaben sind weltweit eine Vielzahl verschiedener Richtlinien in
Gebrauch. Die Wahl des richtigen Schemas ist Geschmacksache -- wichtig ist jedoch,
eine durchdachte und vor allem \emph{konsistente} Verwendung.
Das Literaturverzeichnis ist eine
Zusammenstellung der verwendeten Quellen am \emph{Ende} des Dokuments.
Wichtig ist, dass jeder Literaturverweis im Text einen entsprechenden
Eintrag im Literaturverzeichnis hat und umgekehrt.

\section{Finden von Literator}



\section{Literaturverweise}

\subsection{{\tt cite}}

Für Literaturverweise im laufenden Text verwendet man in \latex die Anweisung
\begin{itemize}
\item[] \verb!\cite{!\textit{Verweise}\verb!}! oder
        \verb!\cite[!\textit{Zusatztext}\verb!]{!\textit{Verweise}\verb!}!.
\end{itemize}

\noindent%
\textit{Verweise} ist eine durch Kommas getrennte Auflistung von Quellen-Schlüsseln
zur Identifikation der entsprechenden Einträge im Literaturverzeichnis.
Mit \textit{Zusatztext} können Ergänzungstexte zum aktuellen Literaturverweis angegeben
werden, wie \zB Kapitel- oder Seitenangaben bei Büchern.
Einige Beispiele dazu:
\begin{itemize}
\item[] \verb!Mehr dazu findet sich in \cite{Kopka98}.! \newline
      $\rightarrow$ Mehr dazu findet sich in \cite{Kopka98}.
\item[] \verb!Mehr über \emph{Styles} in \cite[Kap.\ 3]{Kopka98}.! \newline
      $\rightarrow$ Mehr über \emph{Styles} in \cite[Kap.\ 3]{Kopka98}.
\item[] \verb!Die Angaben in \cite[S.\ 274--277]{Ears99} sind falsch.! \newline
      $\rightarrow$ Die Angaben in \cite[S.\ 274--277]{Ears99} sind aus meiner Sicht falsch.
\item[] \verb!überholt sind auch \cite{Ears99,Microsoft01,Duden97}.! \newline
      $\rightarrow$ überholt sind auch \cite{Ears99,Microsoft01,Duden97}.
\end{itemize}
Die im letzten Beispiel sichtbare automatische Sortierung der Angaben
in der Verweisliste wird übrigens durch die Verwendung des \texttt{cite}-Pakets
erreicht.


\subsection{Häufige Fehler}

\subsubsection{Verweise außerhalb des Satzes}
Literaturverweise sollten innerhalb oder am Ende eines Satzes (vor
dem Punkt) stehen, nicht \emph{außerhalb}, wie
hier. \cite{Oetiker01} $\leftarrow$~FALSCH!

\subsubsection{Zitate}
Falls ein ganzer Absatz (oder mehr) aus einer Quelle zitiert wird,
sollte man den Verweis im vorlaufenden Text platzieren und nicht
\emph{innerhalb} des Zitats selbst. \ZB, die folgenden klaren Worte
aus \cite{Oetiker01}:
\begin{quote}
Typographical design is a craft. Unskilled authors often commit
serious formatting errors by assuming that book design is mostly a
question of aesthetics---``If a document looks good artistically,
it is well designed.'' But as a document has to be read and not
hung up in a picture gallery, the readability and
understandability is of much greater importance than the beautiful
look of it.
\end{quote}
Für das Zitat selbst sollte man übrigens das dafür vorgesehene
%
\begin{itemize}
 \item[] \verb!\begin{quote} ... \end{quote}!
\end{itemize}
%
Environment verwenden, das durch beidseitige Einrückungen das
Zitat vom eigenen Text klar abgrenzt und damit die Gefahr von
Unklarheiten (wo ist das Ende des Zitats?) mindert.
Wenn man möchte, kann man das Innere des Zitats auch in Hochkommas verpacken oder kursiv setzen -- aber nicht beides!


\section{Plagiarismus}

Als "`Plagiat"' bezeichnet man die Darstellung eines fremden Werks als eigene Schöpfung, 
in Teilen oder als Ganzes, egal ob bewusst oder unbewusst.
Plagiarismus ist kein neues Problem im Schulwesen, hat sich aber durch die 
breite Verfügbarkeit elektronischer Quellen in den letzten Jahren dramatisch 
verstärkt und wird keineswegs als Kavaliersdelikt betrachtet.
Viele Schulen bedienen sich als Gegenmaßnahme heute ebenfalls elektronischer Hilfsmittel 
(die den Schülern zum Teil nicht zugänglich sind), und man sollte daher bei
jeder abgegebenen Arbeit damit rechnen, dass sie routinemäßig auf Plagiatsstellen untersucht wird!
Werden solche erst zu einem späteren Zeitpunkt entdeckt, kann das im schlimmsten Fall sogar 
zur nachträglichen (und endgültigen) Aberkennung des Abschlusses führen.

Um derartige Probleme zu vermeiden, sollte man eher übervorsichtig agieren und zumindest folgende Regeln beachten:
%
\begin{itemize}
\item
Die Übernahme kurzer Textpassagen ist nur unter korrekter Quellenangabe zulässig, wobei der Umfang (Beginn und Ende) des Textzitats in jedem einzelnen Fall klar erkenntlich gemacht werden muss. 
\item
Insbesondere ist es nicht zulässig, eine Quelle nur eingangs zu erwähnen und nachfolgend wiederholt nicht-ausgezeichnete Textpassagen als eigene Wortschöpfung zu übernehmen. 
\item
Auf gar keinen Fall tolerierbar ist die direkte oder paraphrierte übernahme längerer Textpassagen, ob mit oder ohne Quellenangabe. Auch indirekt übernommene oder aus einer anderen Sprache übersetzte Passagen müssen mit entsprechenden Quellenangaben gekennzeichnet sein! 
\end{itemize}
%
Im Zweifelsfall findet man detailliertere Regeln in jedem guten Buch über wissenschaftliches Arbeiten oder man fragt sicherheitshalber den Betreuer der Arbeit.



\section{Literaturverzeichnis}

Für die Erstellung des Literaturverzeichnisses gibt es in \latex zwei
Möglichkeiten:
\begin{enumerate}
\item Das Literaturverzeichnis manuell zu formatieren (s.\ \cite[S.\ 56--57]{Kopka98}).
\item Die Verwendung von BibTeX und einer zugehörigen Literaturdatenbank
(s.\ \cite[S.\ 245--255]{Kopka98}).
\end{enumerate}
Tatsächlich ist die erste Variante nur bei sehr wenigen Literaturangaben interessant.
Die Arbeit mit BibTeX macht sich hingegen schnell bezahlt und ist zudem wesentlich
flexibler.

\subsection{Literaturdaten in BibTeX}
\label{sec:bibtex}

BibTeX ist ein eigenständiges Programm, das aus einer "`Literaturdatenbank"' (eine oder mehrere
Textdateien von vorgegebener Struktur) ein für \latex geeignetes Literaturverzeichnis
erzeugt. Dabei ist es möglich, aus einer Reihe von verschiedenen Stilvarianten
(\emph{bibliography styles}) zu wählen.
Literatur zur Verwendung von BibTeX findet man online, \zB in
\cite{Taylor96,Patashnik88}.

Man kann BibTeX-Dateien natürlich mit einem Texteditor manuell erstellen, für
viele Literaturquellen sind auch bereits fertige BibTeX-Einträge im Web verfügbar.
Darüber hinaus gibt es einige Software-Werkzeuge zur Wartung von
BibTeX-Verzeichnissen, zu empfehlen ist beispielsweise
JabRef.%
\footnote{\url{http://jabref.sourceforge.net}}
%und
%\emph{BibEdit}%
%\footnote{\url{www.iui.se/staff/jonasb/bibedit/}}	%% nicht mehr auffindbar!
%von Jonas Björnerstedt.
 
In diesem Dokument wird der BibTeX-Stil \texttt{bababbrv}
("`Babel abbreviated"'),%
\footnote{Aus dem \texttt{babelbib}-Paket von Harald Harders 
(\url{http://www.ctan.org/tex-archive/biblio/bibtex/contrib/babelbib/},
\sa\ Abschn.\ \ref{sec:TippsZuBibtex}).}
eine Variante der Originalversion \texttt{abbrv}, verwendet. Eine weitere
häufig verwendete Stilvariante ist \texttt{babplain} (\bzw\ \texttt{plain}).
Im \latex-Quelltext wird das Literaturverzeichnis am Ende des
Dokuments in dieser Form eingesetzt:
%
\begin{verbatim}
    \bibliographystyle{bababbrv}
    \bibliography{literatur}
\end{verbatim}
%
Die zweite Zeile verweist auf die Literaturdatenbank in der Datei
\url{literatur.bib}, der von BibTeX verarbeitet wird. Unter der Annahme, dass die
Hauptdatei der Arbeit \url{da.tex} heißt, sind zur Erzeugung der Literaturliste
folgende Befehle nötig (im "`Command Prompt"'):
%
\begin{verbatim}
    > latex da
    > bibtex da
    > latex da
    > latex da
\end{verbatim}
%
In der \emph{TeXnicCenter}-Umgebung wird (bei richtiger Einstellung) die
erforderliche BibTeX-Anweisungsfolge automatisch bei jedem \latex-Durchlauf
ausgeführt.




\subsection{Beispiele}
Im Folgenden einige Beispiele für die wichtigsten Formen von Quellenangaben
und die zugehörigen Einträge in der BibTeX-Datei.
Die Form der Literaturangaben ist natürlich abhängig vom verwendeten
BibTeX-Stil, für die hier gezeigten Beispiele wurde \texttt{bababbrv} verwendet.
Weitere Beispiele finden sich im übrigen Text \bzw im
Literaturverzeichnis.


%%------------------------------------------------------
\subsubsection{Buch (\texttt{@book})} 
\nocite{BurgerBurge06}

\begin{description}
\item[\it Elemente:] \hfill\break
     Autor(en), Buchtitel, Verlag, Erscheinungsort, Erscheinungsjahr.
\item[\it Beispiel:] \hfill\break
     {\sc Burger, W.} \btxandshort{.}\ {\sc M.~Burge}: {\em Digitale
  Bildverarbeitung -- Eine Einführung mit Java und ImageJ\/}.
\newblock Springer-Verlag, Heidelberg, 2. \btxeditionshort{.}, 2005.
\item[\it BibTeX-Eintrag:] \mbox{}\par
%
\begin{GenericCode}
@book{BurgerBurge06,
    author={Burger, Wilhelm and Burge, Mark},
    title={Digitale Bildverarbeitung -- 
           Eine Einführung mit Java und ImageJ},
    publisher={Springer-Verlag},
    address={Heidelberg},
    edition={2},
    year={2005},
    language={german}
}
\end{GenericCode}
\end{description}

%%------------------------------------------------------
\subsubsection{Buchbeitrag, Beitrag in einem Sammelband (\texttt{@incollection})}
\nocite{Ears99}
\begin{description}
\item[\it Elemente:] \hfill\break
   Autor(en), Buchtitel, Herausgeber, Buchtitel, [Kapitel,] Verlag, Erscheinungsort,
   Erscheinungsjahr.
\item[\it Beispiel:] \hfill\break
\textsc{Burge, M.} \btxandshort{.}\ \textsc{W.~Burger}: \emph{Ear Biometrics}.
\newblock \Btxinshort{.}\ \textsc{Jain, A.~K.}, \textsc{B.~Ruud} \btxandshort{.}\
  \textsc{P.~Sharath}\ (\btxeditorsshort{.}): \emph{Biometrics: Per\-sonal
  Identification in Networked Society}, \btxchaptershort{.}~13. Klu\-wer
  Academic Publishers, Boston, 1999.
\item[\it BibTeX-Eintrag:] \mbox{}\par
%
\begin{GenericCode}
@incollection{Ears99,
  author={Burge, Mark and Burger, Wilhelm},
  title={Ear Biometrics},
  booktitle={Biometrics: Personal Identification in Networked
	           Society},
  publisher={Kluwer Academic Publishers},
  year={1999},
  address={Boston},
  editor={Jain, Anil K. and Bolle Ruud and Pankanti Sharath},
  chapter={13},
  language={english}
}
\end{GenericCode}
\end{description}

%%------------------------------------------------------
\subsubsection{Konferenzbeitrag, Beitrag in einem Tagungsband (\texttt{@inproceedings})}
\nocite{Burger87}

\begin{description}
\item[\it Elemente:] \hfill\break
  Autor(en), Titel des Beitrags, Titel des Konferenzbands (\emph{Conference
  Proceedings}), Seiten,
  Ort der Tagung, Monat/Jahr der Tagung, Verlag.
\item[\it Beispiel:] \hfill\break
\textsc{Burger, W.} \btxandshort{.}\ \textsc{B.~Bhanu}: \emph{Qualitative Motion
  Understanding}.
\newblock \Btxinshort{.}\ \emph{Proceedings of the Intl.\ Joint Conference on
  Artificial Intelligence}, \btxpagesshort{.}\ 819--821, San Francisco,
  May\ 1987. Morgan Kaufmann Publishers.
\item[\it BibTeX-Eintrag:] \mbox{}\par
%
\begin{GenericCode}
@inproceedings{Burger87,
  author={Burger, Wilhelm and Bhanu, Bir},
  title={Qualitative Motion Understanding},
  booktitle={Proceedings of the Intl.\ Joint Conference
	           on Artificial Intelligence},
  year={1987},
  month=MAY,
  publisher={Morgan Kaufmann Publishers},
  address={San Francisco},
  pages={819-821},
  language={english}
}
\end{GenericCode}
\end{description}

%%------------------------------------------------------
\subsubsection{Zeitschriftenbeitrag, Journal Paper (\texttt{@article})}
\nocite{Guttman01}

\begin{description}

\item[\it Elemente:] \hfill\break
    Autor(en), Titel des Beitrags, Zeitschrift, Band, Seiten, Monat(e),
    Erscheinungsjahr.
\item[\it Beispiel:] \hfill\break
\textsc{Guttman, E.}: \emph{Autoconfiguration for {IP} Networking}.
\newblock IEEE Internet Computing, 5:81--86, Mai/Juni\ 2001.
\item[\it BibTeX-Eintrag:] \mbox{}\par
%
\begin{GenericCode}
@article{Guttman01,
  author={Guttman, Erik},
  title={Autoconfiguration for {IP} Networking},
  journal={IEEE Internet Computing},
  volume={5},
  year={2001},
  pages={81-86},
  month=MAY # "/" # JUN,
  language={english}
}
\end{GenericCode}
\end{description}

%%------------------------------------------------------
\subsubsection{Dissertation (\texttt{@phdthesis})}
\nocite{Eberl87}

\begin{description}
\item[\it Elemente:] \hfill\break
  Autor, Titel, Hochschule, Institut, Adresse, Monat, Jahr.
\item[\it Beispiel:] \hfill\break
\textsc{Eberl, G.}: \emph{Automatischer Landeanflug durch Rechnersehen}.
\newblock \btxphdthesis{}, Universität der Bundeswehr, Fakultät für
  Raum- und Luftfahrttechnik, München, August\ 1987.
\item[\it BibTeX-Eintrag:] \mbox{}\par
%
\begin{GenericCode}
@phdthesis{Eberl87,
  author={Eberl, Gerhard},
  title={Automatischer Landeanflug durch Rechnersehen},
  school={Universität der Bundeswehr, Fakultät für
	        Raum- und Luftfahrttechnik},
  year={1987},
  month=AUG,
  address={München},
  language={german}
}
\end{GenericCode}
\end{description}

%%------------------------------------------------------
\subsubsection{Diplomarbeit (\texttt{@mastersthesis})}
\nocite{Wintersberger00}

\begin{description}
\item[\it Elemente:] \hfill\break
  Autor, Titel, Hochschule, Institut, Adresse, Monat, Jahr.
\item[\it Beispiel:] \hfill\break
\textsc{Wintersberger, M.}: \emph{Realisierung eines Internet-Auktionshauses
  basierend auf {PHP} und {MySQL}}.
\newblock \btxmastthesis{}, Fachhochschule Hagenberg, Medientechnik und
  -design, Hagenberg, Austria, Juni\ 2000.
\item[\it BibTeX-Eintrag:] \mbox{}\par
%
\begin{GenericCode}
@mastersthesis{Wintersberger00,
  author={Wintersberger, Markus},
  title={Realisierung eines Internet-Auktionshauses
	       basierend auf {PHP} und {MySQL}},
  school={Fachhochschule Hagenberg, Medientechnik und -design},
  year={2000},
  month=JUN,
  address={Hagenberg, Austria},
  language={german}
}
\end{GenericCode}
\end{description}

%%------------------------------------------------------


\subsubsection{Bachelorarbeit (\texttt{@masterthesis} modifiziert)}

Bachelorarbeiten gelten in der Regel zwar nicht als "`richtige"' Publikationen, bei Bedarf muss man sie aber doch referenzieren können. Das geht ebenfalls mit dem Eintrag \verb!@mastersthesis{}!, versehen mit einem entsprechenden \verb!type!-Feld, zum Beispiel \cite{Bacher04}:
%
\begin{description}
\item[\it BibTeX-Eintrag:] \mbox{}\par
%
\begin{GenericCode}
@mastersthesis{Bacher04,
  author={Bacher, Florian},
  title={Interaktionsmöglichkeiten mit Bildschirmen 
         und gro\ss flächigen Projektionen},
  school={Fachhochschule Hagenberg, Medientechnik und -design},
  year={2004},
  month=JUN,
  address={Hagenberg, Austria},
  type={Bachelorarbeit},
  url={http://theses.fh-hagenberg.at},
  language={german}
}
\end{GenericCode}
\end{description}
%
Man beachte, dass der Inhalt des Eintrags \verb!url={..}! automatisch und ohne weitere Auszeichnung mit den Einstellungen des \verb!\url{}! Makros gesetzt wird.

%%------------------------------------------------------
\subsubsection{Bericht, Technical Report (\texttt{@techreport})}
\nocite{Beeler48}

Das sind typischerweise nummerierte Berichte aus Unternehmen, Hochschulinstituten oder Forschungsprojekten, 
Wichtig ist, dass die herausgebende Organisationseinheit (Institut, Fakultät etc.) und 
Adresse vollständig angegeben wird.
%
\begin{description}
\item[\it Elemente:] \hfill\break
  Autor(en), Titel, Report-Nummer, Institution, Adresse, Monat, Jahr.
\item[\it Beispiel:] \hfill\break
\textsc{Beeler, D.~E.} \btxandshort{.}\ \textsc{J.~P. Mayer}: \emph{Measurement of the
  wing and tail loads during acceptance test of the {Bell} {XS-1} research
  airplane}.
  \newblock \Btxtechrepshort{.}\ NACA-RM-L7L12, NASA Dryden Flight Research
  Center, Edwards, CA, April\ 1948. \url{www.dfrc.nasa.gov/DTRS/1948/index.html}
\item[\it BibTeX-Eintrag:] \mbox{}\par
%
\begin{GenericCode}
@techreport{Beeler48,
  author={Beeler, De E. and Mayer, John P.},
  title={Measurement of the wing and tail loads 
         during acceptance test of the {Bell} 
         {XS-1} research airplane},
  institution={NASA Dryden Flight Research Center},
  year={1948},
  address={Edwards, CA},
  number={NACA-RM-L7L12},
  url={www.dfrc.nasa.gov/DTRS/1948/index.html},
  language={english}
}
\end{GenericCode}
\end{description}

%%------------------------------------------------------
\subsubsection{Handbuch, Manual (\texttt{@manual})}

Dieser Publikationstyp bietet sich zum einen natürlich für Handbücher und Produktbeschreibungen an, insbesondere aber auch für viele Dokumente, bei denen etwa kein Autor genannt ist, wie bei vielen Online-Quellen.

\begin{description}
\item[\it Elemente:] \hfill\break
  Autor(en) oder Institution, Titel des Dokuments, Adresse, Datum.
\item[\it Beispiel:] \hfill\break
{\sc Microsoft Corporation}: {\em Application Service Providers: Evolution
  and Resources\/}, Jan.\ 2001.
\newblock \url{www.microsoft.com/ISN/downloads/ASP.doc}, Kopie auf CD-ROM.
\item[\it BibTeX-Eintrag:] \mbox{}\par
%
\begin{GenericCode}
@manual{Microsoft01,
  organization={Microsoft Corporation},
  title={Application Service Providers: Evolution and Resources},
  year={2001},
  month=JAN,
  url={www.microsoft.com/ISN/downloads/ASP.doc}, 
  note={Kopie auf CD-ROM},
  language={english}
}
\end{GenericCode}
\end{description}


%%------------------------------------------------------
\subsubsection{Sonderfälle (\texttt{@misc})}

Es kommt immer wieder vor, dass spezielle
Quellen zu berücksichtigen sind. Bei Technischen Diplomarbeiten fallen
vor allem Normen und Patente in diesen Bereich. Andere Beispiele sind
Filme, Videos oder Musikstücke. Für diese gibt es in BibTeX leider keine eigenen Vorkehrungen, sie können aber mit dem \texttt{@misc}-Eintrag
erstellt werden. Nachfolgend einige Beispiele für häufig
erforderliche Situationen.

\paragraph{Patent:} In Patenten befindet sich ein großer Teil des aktuellsten
Technischen Wissens. Dementsprechend sind in technischen Diplomarbeiten häufig
Patente als Quellen anzuführen. Als Beispiel dient hier das European Patent No.
EP1765031 \cite{PAT07}

%
\begin{description}
\item[\it BibTeX-Eintrag:] \mbox{} \par 
\begin{GenericCode}
@misc {PAT07,
    author = {M. Anisetti and {C.A.} Ardagna and V. Bellandi and E. Damiani},
    howpublished = {European Patent No. EP1765031},
    title = {Method, System, Network and Computer Program Product for Positioning in a Mobile Communications Network},
    month = MAR,
    day = {21},
    year = {2007},
    language={english}
} 
\end{GenericCode}
\end{description}

\paragraph{Norm:} Normen sind in der Technik allgegenwärtig. Beispiele sind zwei
Teile der Norm DIN 9241. Teil 11\cite{DIN9241-11} und Teil
110\cite{DIN9241-110}.

\begin{description}
\item[\it BibTeX-Eintrag:] \mbox{} \par 
\begin{GenericCode}
@misc{DIN9241-110,
  publisher = {Europäisches Kommittee für Normung},
  title = {Ergonomie der Mensch-System-Interaktion: Teil 110 Grundsätze der Dialoggestaltung},
  howpublished = {DIN 9241-110},
  year = 2006,
  language={german}
}

@misc{DIN9241-11,
  howpublished = {DIN EN ISO 9241-11},
  title = {Ergonomische Anforderungen für Bürotätigkeiten mit Bildschirmgeräten 
Teil 11: Anforderungen an die Gebrauchstauglichkeit - Leitsätze},
  year = 1999,
  month =JAN,
  language={german}
}
\end{GenericCode}
\end{description}

%

\paragraph{Audio-CD:} Hier ein Vorschlag unter der Annahme, dass es sich um das Werk eines oder mehrerer klar identifizierbarer Künstler (Autoren) handelt. Ansonsten könnte man alternativ ein Format ähnlich wie für einen Film (\su) verwenden.\nocite{Zappa95}
%
\begin{description}
\item[\it Beispiel:] \hfill\break
\textsc{Zappa, F.}: \emph{Freak Out}.
\newblock Audio-CD, Rykodisc, New York, \btxmonmayshort{.}\ 1995.
%
\item[\it BibTeX-Eintrag:] \mbox{} \par 
\begin{GenericCode}
@misc{Zappa95,
  author={Zappa, Frank},
  title={Freak Out},
  year={1995},
  month=MAY,
  howpublished={Audio-CD, Rykodisc, New York},
  language={english}
}
\end{GenericCode}
\end{description}


\paragraph{Film/DVD:} In diesem Fall wird bewusst \emph{kein} Autor angegeben, weil dieser bei einem Film \ia\ nicht eindeutig zu benennen ist. Aus diesem Grund wird ein \texttt{key}-Feld angegeben, das (anstelle des Autors) für die Sortierung verwendet wird.\nocite{Nosferatu}
%
\begin{description}
\item[\it Beispiel:] \hfill\break
\emph{Nosferatu -- A Symphony of Horrors}.
\newblock Drehbuch/Regie: F. W. Murnau. Mit Max Schreck, Gustav von Wangenheim,
  Greta Schröder, 1922.
\newblock DVD, Film Preservation Associates, London (1991).
\item[\it BibTeX-Eintrag:] \mbox{}\par
\begin{GenericCode}
@misc{Nosferatu,
  key={Nosferatu},
  title={Nosferatu -- A Symphony of Horrors},
  year={1922},
  howpublished={Drehbuch/Regie: F. W. Murnau. 
  Mit Max Schreck, Gustav von Wangenheim, Greta Schröder},
  note={DVD, Film Preservation Associates, London (1991)},
  language={german}
}
\end{GenericCode}
\end{description}
Ein weiteres Filmbeispiel ist \emph{Psycho} von Alfred Hitchcock \cite{Psycho}.


\paragraph{Persönliche Kommunikation:}
Ein Sonderfall sind Quellen ohne
irgendeiner Form von Veröffentlichung, wie \zB Hinweise aus
persönlichen Gesprächen (\emph{personal communication}) mit Fachleuten.
Obwohl man auf diese Art von Verweisen \textbf{nur im äußersten Fall}
zurückgreifen sollte, hier ein Beispiel wie man damit umgeht:
\nocite{Kreisky75}

\begin{description}
\item[\it Beispiel:] \hfill\break
\textsc{Kreisky, B.}: \emph{Kaffeehausgespräch zur Lage der Nation}.
\newblock Persönl.\ Gespräch, Nov.\ 1975.
\item[\it BibTeX-Eintrag:] \mbox{}\par
%
\begin{GenericCode}
@misc{Kreisky75,
  author={Kreisky, Bruno},
  title={Kaffeehausgespräch zur Lage der Nation},
  year={1975},
  month=NOV,
  howpublished={Persönl.\ Gespräch},
  language={german}
}
\end{GenericCode}
\end{description}

%%------------------------------------------------------
\subsubsection{Online-Quellen, Wiki-Einträge etc.}
\label{sec:OnlneQuellen}
\nocite{Microsoft01}

Verweise auf Webseiten sollten generell nur in Ausnahmefällen
verwendet werden und auch nur dann, wenn keine entsprechende
andere Publikation verfügbar ist und sich an der angegeben Adresse
(URL) auch tatsächlich ein Dokument befindet. Bei Online-Quellen 
\emph{ohne explizit angegebenem Autor}, \zB Firmen-Homepages, 
Link-Sammlungen und \va\ auch \emph{Wikipedia}-Seiten, sollte 
man die entsprechende Webadresse \emph{nicht} in die
Literaturliste aufnehmen sondern direkt im Text als \textbf{Fußnote} 
angeben. Beispielsweise bezeichnet man als "`Reliquienschrein"' einen 
Schrein, in dem die Reliquien eines oder mehrerer Heiliger aufbewahrt werden.%
\footnote{\url{http://de.wikipedia.org/wiki/Reliquienschrein}}

Wird der Diplomarbeit ein elektronischer Datenträger (CD-ROM, DVD
etc.) beigelegt, empfiehlt sich die angeführten Webseiten in
elektronischer Form (vorzugsweise als PDF-Da\-tei\-en) abzulegen
und mit einem entsprechenden Verweis im Literaturverzeichnis
("`Kopie auf CD-ROM"') zu versehen.

\subsection{Tipps zur Erstellung von BibTeX-Dateien}
\label{sec:TippsZuBibtex}

\subsubsection{\texttt{month}-Attribut}

Für die Angabe des \texttt{month}-Attributs sollte man grundsätzlich die zwölf in BibTeX bereits vordefinierten Abkürzungen
\begin{quote}
\texttt{JAN}, \texttt{FEB}, \texttt{MAR}, \texttt{APR}, 
\texttt{MAY}, \texttt{JUN}, \texttt{JUL}, \texttt{AUG}, 
\texttt{SEP}, \texttt{OCT}, \texttt{NOV}, \texttt{DEC}
\end{quote}
verwenden, und zwar \emph{ohne} die sonst erforderlichen Klammern (oder Hochkommas), also beispielsweise einfach mit
% Siehe auch Q13 in http://www.ctan.org/tex-archive/biblio/bibtex/contrib/doc/btxFAQ.pdf
%
\begin{quote}
\verb!month=AUG!
\end{quote}
%
Der richtige Monatsname wird abhängig von der Spracheinstellung für das Dokument automatisch eingesetzt.
Ein Intervall über \emph{mehrere} Monate kann man in der (etwas eigenartigen) BibTeX-Syntax so angeben (verwendet \zB\ für "`Mai/Juni"' in \cite{Guttman01}):
\begin{quote}
\verb!month=MAY # "/" # JUN!
\end{quote}


\subsubsection{\texttt{language}-Attribut}

Das in dieser Vorlage verwendete \verb!babelbib!-Paket ermöglicht den korrekten Satz mehrsprachiger Literaturverzeichnisse. Dazu ist es ratsam, bei jedem Quelleneintrag auch die enstsprechende Sprache anzugeben, also beispielsweise
\begin{quote}
\verb!language={german}! \quad oder \quad \verb!language={english}!
\end{quote}
für ein deutsch- \bzw\ englischsprachiges Dokument.

\subsubsection{\texttt{edition}-Attribut}

Die Verwendung von \verb!babelbib! lässt auch die früheren Probleme mit der
Angabe der Auflagennummer von Büchern. Nunmehr ist lediglich die Nummer selbst anzugeben, also etwa
\begin{quote}
\verb!edition={3}!
\end{quote}
bei einer dritten Auflage. Die richtige Punktuation in der Quellenangabe wird in Abhängigkeit von der Spracheinstellung durch \texttt{babelbib} automatisch hinzugefügt ("`3.\ Auflage"' \bzw\ "`3rd edition"').


\subsubsection{Listing aller Quellen}

Durch die Anweisung \verb!\nocite{*}! -- an beliebiger Stelle im Dokument platziert -- werden \emph{alle} bestehenden Einträge der BibTeX-Datei im Literaturverzeichnis aufgelistet, also auch jene, für die es keine explizite \verb!\cite{}! Anweisung gibt. Das ist ganz nützlich, um während des Schreibens der Arbeit eine aktuelle Übersicht auszugeben. Normalerweise müssen aber alle angeführten Quellen auch im Text referenziert sein!
